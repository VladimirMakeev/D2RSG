\subsubsection{Зона}
\label{zone}
Определяет зону на карте сценария.

Обязательные поля:
\begin{itemize}
\item \texttt{id} - идентификатор зоны. Число, уникально определяющее зону
\item \texttt{type} - тип зоны. См. \hyperref[zoneTypes]{\selectlanguage{Russian}Типы зон}
\item \texttt{size} - относительный размер зоны
\end{itemize}

В случае если в качестве типа зоны выбрана стартовая локация \texttt{type = Zone.PlayerStart}, нужно указать уникальную расу для игрока в этой зоне:
\begin{itemize}
\item \texttt{race} - раса игрока. Одна из поданного в \texttt{getContents} списка рас или заранее заданная. См. \hyperref[raceTypes]{\selectlanguage{Russian}Расы}
\end{itemize}

Необязательные поля:
\begin{itemize}
\item \texttt{mines} - источники ресурсов в зоне. См. \hyperref[crystals]{\selectlanguage{Russian}Источники ресурсов}
\item \texttt{towns} - список нейтральных городов. См. \hyperref[city]{\selectlanguage{Russian}Город}
\item \texttt{ruins} - список руин. См. \hyperref[ruin]{\selectlanguage{Russian}Руины}
\item \texttt{merchants} - список торговцев. См. \hyperref[merchant]{\selectlanguage{Russian}Торговец}
\item \texttt{mages} - список башен магов. См. \hyperref[mage]{\selectlanguage{Russian}Башня мага}
\item \texttt{mercenaries} - список лагерей наемников. См. \hyperref[mercenary]{\selectlanguage{Russian}Лагерь наемников}
\item \texttt{trainers} - список тренировочных лагерей. См. \hyperref[trainer]{\selectlanguage{Russian}Тренер}
\item \texttt{stacks} - список групп нейтральных отрядов в зоне. См. \hyperref[neutralStacks]{\selectlanguage{Russian}Нейтральные отряды}
\item \texttt{bags} - неохраняемые сундуки в зоне. См. \hyperref[bags]{\selectlanguage{Russian}Неохраняемые сундуки}
\end{itemize}

Для стартовых локаций игроков также возможно задание гарнизона, предметов в столице и заклинаний известных игроку со старта:
\begin{itemize}
\item \texttt{capital} - столица игрока. См. \hyperref[capital]{\selectlanguage{Russian}Столица}
\end{itemize}

Примеры:\\
Стартовая зона с относительным размером 20. Здесь будет столица первого игрока из списка рас.
В зоне будут созданы 1 золотой рудник и 1 источник маны рун.

\begin{figure}[H]
\lstinputlisting{docExamples/zoneExample1.lua}
\end{figure}

Стартовая зона с относительным размером 20. Здесь будет столица первого игрока из списка рас.
Независимо от выбранной расы, в гарнизоне столице будут созданы юниты империи общей ценностью 150 - 175.
В инвентаре столицы будут созданы зелья лечения и воскрешения ценностью 750 - 800.
Игроку-владельцу со старта будут известны заклинания 'Ледяной щит' и 'Молния'.

\begin{figure}[H]
\lstinputlisting{docExamples/zoneExample2.lua}
\end{figure}

Обычная зона (сокровищница) с относительным размером 35.
В зоне будут случайно расположены 12 нейтральных отрядов общей ценностью 1200 - 1300,
а также 5 нейтральных отрядов общей ценностью 2000.

\begin{figure}[H]
\lstinputlisting{docExamples/zoneExample3.lua}
\end{figure}