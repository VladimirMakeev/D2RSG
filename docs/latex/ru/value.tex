\subsubsection{Ценность}
\label{value}
Задает ценность группы (отряда), юнита, предмета или заклинания в диапазоне \texttt{[min : max]}.

Расчет ценности генератором сценариев:
\begin{itemize}
\item ценность групп, отрядов и юнитов определяется их опытом за убийство, значение \texttt{XP\_KILLED} из GUnits.dbf
\item ценность предметов определяется как сумма всех ресурсов в стоимости их покупки у торговца, значение \texttt{VALUE} из GItems.dbf
\item ценность заклинаний определяется как сумма всех ресурсов в стоимости их покупки в башне мага, значение \texttt{BUY\_C} из GSpells.dbf
\end{itemize}

Сумма всех ресурсов:\\
К примеру предмет \texttt{А} стоит 100 золота и 50 маны жизни. Сумма всех ресурсов его стоимости даст ценность равную 150.
Предмет \texttt{Б} стоит 150 золота. Его ценность, также как у \texttt{А}, будет 150.\\
Для генератора золото и все типы маны равноценны.

Обязательные поля:
\begin{itemize}
\item \texttt{min} - минимальное значение ценности. Целое число
\item \texttt{max} - максимальное значение ценности. Целое число
\end{itemize}

Примеры:\\
Ценность от 100 до 200 включительно:\\
\begin{lstlisting}
{ min = 100, max = 200 }
\end{lstlisting}
Ценность 175:\\
\begin{lstlisting}
{ min = 175, max = 175 }
\end{lstlisting}