\subsubsection{Дипломатия}
\label{diplomacy}
Определяет дипломатические отношения между двумя расами.

Обязательные поля:
\begin{itemize}
\item \texttt{raceA} - первая раса. См. \hyperref[raceTypes]{\selectlanguage{Russian}Раса}
\item \texttt{raceB} - вторая раса. См. \hyperref[raceTypes]{\selectlanguage{Russian}Раса}
\item \texttt{relation} - уровень отношений в диапазоне \texttt{[0 : 100]}. 0 - вражда, 100 - мир
\end{itemize}
Необязательные поля:
\begin{itemize}
\item \texttt{alliance} - \texttt{true} если расы находятся в союзе. Не может быть \texttt{true} когда \texttt{alwaysAtWar} также \texttt{true}. По умолчанию \texttt{false}
\item \texttt{alwaysAtWar} - \texttt{true} если расы находятся в состоянии вечной вражды. Не может быть \texttt{true} когда \texttt{alliance} также \texttt{true}. По умолчанию \texttt{false}
\item \texttt{permanentAlliance} - \texttt{true} если альянс считается постоянным для ИИ. Не может быть \texttt{true} когда \texttt{alliance} \texttt{false}. По умолчанию \texttt{false}
\end{itemize}

Примеры:\\
Две расы находятся в состоянии вражды:

\begin{figure}[H]
\lstinputlisting{docExamples/diplomacyExample1.lua}
\end{figure}

Две расы находятся в состоянии вражды, заключение союза невозможно:

\begin{figure}[H]
\lstinputlisting{docExamples/diplomacyExample2.lua}
\end{figure}

Две расы находятся в союзе. ИИ не может расторгнуть союз:

\begin{figure}[H]
\lstinputlisting{docExamples/diplomacyExample3.lua}
\end{figure}