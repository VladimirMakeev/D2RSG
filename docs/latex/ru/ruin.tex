\subsubsection{Руины}
\label{ruin}
Описывает руины, их защитников и награду.

Обязательные поля:
\begin{itemize}
\item \texttt{guard} - юниты внутри руин. См. \hyperref[group]{\selectlanguage{Russian}Группа / Отряд}. \textbf{\selectlanguage{Russian}Важно:} предметы указанные для этой группы игнорируются!
\end{itemize}

Необязательные поля:
\begin{itemize}
\item \texttt{gold} - награда за руины в золоте. См. \hyperref[value]{\selectlanguage{Russian}Ценность}
\item \texttt{loot} - предмет-награда за руины. См. \hyperref[loot]{\selectlanguage{Russian}Награда}. \textbf{\selectlanguage{Russian}Важно:} только один предмет из награды будет создан. Обязательные предметы имеют приоритет над случайными. В случае обязательных предметов будет выбран первый из списка.
\end{itemize}

Пример:\\
Руины с наградой из эликсира восстановления и 200 - 250 золота.
Охраняются группой защитников империи или нейтральных людей

\begin{figure}[H]
\lstinputlisting{docExamples/ruinExample.lua}
\end{figure}