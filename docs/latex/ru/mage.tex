\subsubsection{Башня мага}
\label{mage}
Описывает башню мага, его заклинания и потенциальную охрану. Заклинания делятся на случайные и обязательные.

Обязательных полей нет, пустая таблица описывает башню мага без заклинаний и охраны.

Необязательные поля:
\begin{itemize}
\item \texttt{value} - общая ценность случайных заклинаний. Ценностью заклинания считается стоимость его покупки (сумма всех ресурсов), поле \texttt{BUY\_C} из GSpells.dbf.  См. \hyperref[value]{\selectlanguage{Russian}Ценность}
\item \texttt{spellLevel} - диапазон уровней каждого случайного заклинания. Допустимые значения: \texttt{[1 : 5]}. Если не задано, в башне могут продаваться заклинания любых уровней
\item \texttt{spellTypes} - список типов случайных заклинаний которые могут появиться. См. \hyperref[spellTypes]{\selectlanguage{Russian}Типы заклинаний}. В случае пустого списка могут быть выбраны заклинания любых типов
\item \texttt{spells} - список идентификаторов заклинаний из GSpells.dbf которые обязательно появятся в башне мага.
\item \texttt{guard} - отряд охраняющий вход в башню мага. См. \hyperref[group]{\selectlanguage{Russian}Группа / Отряд}
\item \texttt{aiPriority} - приоритет башни мага для ИИ в диапазоне \texttt{[0 : 6]}
\end{itemize}

Пример:\\
Башня мага с заклинаниями лечения и ускорений общей ценностью от 200 до 1000.
В ассортименте обязательно появятся 'Быстрота' и 'Молния'

\begin{figure}[H]
\lstinputlisting{docExamples/mageExample1.lua}
\end{figure}

Башня мага с заклинаниями только 2 уровня общей ценностью от 2700 до 3000.
Если каждое заклинание 2 уровня стоит порядка 300 золота (ценность 300),
получим примерно 9-10 заклинаний в ассортименте

\begin{figure}[H]
\lstinputlisting{docExamples/mageExample2.lua}
\end{figure}