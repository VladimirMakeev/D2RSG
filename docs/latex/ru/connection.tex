\subsubsection{Проход}
\label{connection}
Определяет соединение двух зон, проход между ними.
Между двумя зонами может быть более одного прохода.

Обязательные поля:
\begin{itemize}
\item \texttt{from} - идентификатор первой зоны. См. \hyperref[zone]{\selectlanguage{Russian}Зона}
\item \texttt{to} - идентификатор второй зоны. См. \hyperref[zone]{\selectlanguage{Russian}Зона}
\end{itemize}

Необязательные поля:
\begin{itemize}
\item \texttt{guard} - отряд охраняющий проход между зонами. См. \hyperref[group]{\selectlanguage{Russian}Группа / Отряд}
\end{itemize}

Примеры:\\
Проход между зонами 0 и 1, без охраны
\begin{figure}[h]
\lstinputlisting{docExamples/connectionExample1.lua}
\end{figure}

Проход между зонами 1 и 0, аналогичен предыдущему примеру
\begin{figure}[h]
\lstinputlisting{docExamples/connectionExample2.lua}
\end{figure}

Проход между зонами 0 и 3.
Охраняется отрядом ценностью 1750 - 2250.
В инвентаре отряда будут созданы артефакты, знамена или перманентные зелья общей ценностью 2500 - 3000
\begin{figure}[h]
\lstinputlisting{docExamples/connectionExample3.lua}
\end{figure}