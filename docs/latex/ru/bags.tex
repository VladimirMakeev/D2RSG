\subsubsection{Неохраняемые сундуки}
\label{bags}
Описывает неохраняемые сундуки внутри зоны.

Обязательных полей нет. Пустая таблица описывает зону без сундуков.

Необязательные поля:
\begin{itemize}
\item \texttt{loot} - общая награда всех сундуков. Будет равномерно распределена по всем сундукам. См. \hyperref[loot]{\selectlanguage{Russian}Награда}
\item \texttt{count} - количество сундуков в зоне
\item \texttt{aiPriority} - приоритет сундуков для ИИ в диапазоне \texttt{[0 : 6]}
\end{itemize}

Пример:\\
10 сундуков с зельями на буст и воскрешение общей суммой 5000 - 7500.
На 10 сундуков обязательно будут созданы 15 эликсиров восстановления.

\begin{figure}[H]
\lstinputlisting{docExamples/bagsExample.lua}
\end{figure}