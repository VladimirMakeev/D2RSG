\subsubsection{Группа / Отряд}
\label{group}
Описывает группу юнитов в руинах или гарнизоне города, либо отряд с лидером.

Обязательные поля:
\begin{itemize}
\item \texttt{value} - общая ценность юнитов. См. \hyperref[value]{\selectlanguage{Russian}Ценность}
\end{itemize}

Необязательные поля:
\begin{itemize}
\item \texttt{subraceTypes} - список сабрас определяющий какие типы юнитов могут быть созданы. См. \hyperref[subraceTypes]{\selectlanguage{Russian}Субрасы}. В случае пустого списка могут быть созданы юниты любых субрас
\item \texttt{loot} - награда. См. \hyperref[loot]{\selectlanguage{Russian}Награда}. В случае отряда награда создаст предметы инвентаря. Для группы в гарнизоне города награда создаст предметы в городе. \textbf{\selectlanguage{Russian}Важно:} награда группы внутри руин \textbf{\selectlanguage{Russian}не учитывается!}. Награда руин задается отдельно. См. \hyperref[ruin]{\selectlanguage{Russian}Руины}
\end{itemize}

Пример:\\
Группа / Отряд из гномов или нейтральных людей.
В качестве награды будут созданы зелья лечения и бафов на 200 - 300 ценности.

\begin{figure}[H]
\lstinputlisting{docExamples/groupExample.lua}
\end{figure}