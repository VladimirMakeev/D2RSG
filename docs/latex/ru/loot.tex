\subsubsection{Награда}
\label{loot}
Описывает награду (лут). Награда может состоять из случайных и обязательных предметов.

Обязательных полей нет, пустая таблица создаст пустую награду.

Необязательные поля:
\begin{itemize}
\item \texttt{value} - ценность случайных предметов в награде. См. \hyperref[value]{\selectlanguage{Russian}Ценность}
\item \texttt{itemValue} - диапазон ценности каждого случайного предмета в награде. См. \hyperref[value]{\selectlanguage{Russian}Ценность}
\item \texttt{itemTypes} - список типов случайных предметов которые могут быть созданы. В случае пустого списка может быть создан предмет любого типа, кроме специального \texttt{Item.Special}
\item \texttt{items} - список предметов которые обязательно должны быть созданы. Не зависит от ценности награды. См. \hyperref[item]{\selectlanguage{Russian}Предмет награды}
\end{itemize}

Примеры:\\
Случайная награда из зелий-бафов и воскрешения общей стоимостью от 500 до 750,
а также от 3 до 5 эликсиров восстановления

\begin{figure}[H]
\lstinputlisting{docExamples/lootExample1.lua}
\end{figure}

Награда из 5 зелий восстановления

\begin{figure}[H]
\lstinputlisting{docExamples/lootExample2.lua}
\end{figure}

Случайная награда из знамен и сапог общей стоимостью от 2500 до 2750

\begin{figure}[H]
\lstinputlisting{docExamples/lootExample3.lua}
\end{figure}

Случайная награда общей стоимостью от 1600 до 1800 
из зелий лечения или воскрешения каждое стоимостью от 200 до 450. 
Подняв минимальную ценность каждой случайной вещи до 200 
мы исключаем дешевые зелья лечения (50 hp) из награды.

\begin{figure}[H]
\lstinputlisting{docExamples/lootExample4.lua}
\end{figure}